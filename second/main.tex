\documentclass[12pt, a4paper, oneside]{memoir}
%%% Fonts and language setup.
\usepackage{polyglossia}
% Setup fonts.
\usepackage{fontspec}
\usepackage{amsmath}
\usepackage{listings}
\usepackage{fontspec}
\usepackage{amsmath, amsfonts, amssymb, amsthm, mathtools}

\newfontface\lserif{Liberation Serif}

\newcommand{\Csh}{C{\lserif\#}}
\setmainfont{CMU Serif}
\setsansfont{CMU Sans Serif}
\setmonofont{CMU Typewriter Text}

\NewDocumentEnvironment{solution}{mmm +b}{
    \textbf{ДАНО:} #1

    \textbf{НАЙТИ:} #2

    \textbf{МЕТОД РЕШЕНИЯ:} #3

    \textbf{РЕШЕНИЕ:} #4
}{}

\newcommand{\romannumeralCaps}[1]{\uppercase\expandafter{\romannumeral #1\relax}}

\setdefaultlanguage{russian}
\setotherlanguage{english}

\title{Теория формальных языков и трансляций. Домашняя работа №2}
\author{Кубышкин Е.А., группа 21.Б-07}
\date{22 сентября 2023г.}

\begin{document}

\maketitle

\section*{Упражнение \romannumeralCaps 1-2.1}
\begin{solution}
    {Грамматика $G = (V_N, V_T, P, S)$, где $V_N = \{S, A, B\}, \\ V_T = \{0, 1\}$,
        \begin{align*}
            P = \{
             & S \to 0A, \   S \to 1B, \   S \to 0,    \\
             & A \to 0A,\    A \to 0S,\    A \to 1B,   \\
             & B\to 1B, \    B\to 1,   \    B\to 0 \}.
        \end{align*}
    }
    {Язык, порождающийся данной грамматикой.}
    {Логические рассуждения.}
    \begin{enumerate}
        \item Понятно, что если мы приходим в нетерминал $B$, то мы обязаны накапливать единички и завершаться либо $0$, либо $1$. Таким образом, вне зависимости от того, какой был префикс $\mathrm{pref}$, если был хотя бы один нетерминал $B$, то слово имеет вид: $\mathrm{pref}\ 1^+ 0^k$, где $k \in \{0, 1\}$
        \item Далее рассмотрим $\mathrm{pref}$ (уже отсекая любые пути, содержащие $B$):
              \begin{enumerate}
                  \item Можем сразу прийти в вариант с $B$, так что возможно, что $\mathrm{pref} = \varepsilon$
                  \item Далее единственный вариант~--- идти в $0A$. Отсюда мы можем накопить любое количество $0$, из-за правила $A\to 0A$, а после варианты $A \to 0S$ и $A\to 1B$ по сути не отличаются, потому что мы либо ничего не меняем и продолжаем накапливать $1$, либо зайдём в $B$, тем самым закончив разбирать $\mathrm{pref}$. Значит, $\mathrm{pref} = 0^*$
              \end{enumerate}
        \item Если вообще не идти в $B$:
              \begin{enumerate}
                  \item Можно сразу пойти в $0$
                  \item Можно пойти в $0A$, тогда самый короткий способ закончить слово: $S \to 0A \to 00S \to 000$. Так что 2 нулей быть не может, только $1, 3$ или больше.
                  \item Или можно продолжать любое количество $0$, тогда слово будет иметь вид $0^k$, где $k \in \mathbb{N} \setminus \{2\}$
              \end{enumerate}
    \end{enumerate}
    Подводя итог, язык, порождающийся грамматикой имеет вид
    $\{0^k\} \cup \{0^*1^+0^b\}$, где $b \in \{0,1\}, k\in \mathbb{N} \setminus \{2\}$
\end{solution}
\section*{Упражнение \romannumeralCaps 1-2.2}
\begin{solution}
    {Язык $L = \{w \mid w \in \{0,1\}^* \}$, где $w$ не содержит двух последовательных единиц.}
    {Регулярную грамматику, порождающую этот язык.}
    {}
    \begin{align*}
         & G = \{V_N, V_T, P, S\}, \intertext{где}
         & V_N = \{A, B, S\}, V_T = \{0, 1\},               \\
         & P = \{ S \to 0B, \ S\to 1A, \ S \to 1, \ S \to 0 \\
         & A\to 0B, \ A\to 0                                \\
         & B\to 0B, \ B\to 1A, \ B \to 1\}
    \end{align*}
\end{solution}
\section*{Упражнение \romannumeralCaps 1-2.3}
\begin{solution}
    {контекстно-свободная грамматика $G = (\{S, A\}, \{a, b\}, P, S)$, где $P = \{S\to aAS, \ S \to a, \ A \to SbA, \ A \to ba, \ A \to SS\}$}
    {Неформально описать слова, порождающиеся этой грамматикой}
    {Логические рассуждения}


\end{solution}
\section*{Упражнение \romannumeralCaps 1-2.4}
\begin{solution}
    {Алгоритм теоремы 2.2}
    {Проверить принадлежность слов $(abaa),\ (abbb),\ (baaba) $ к грамматике из примера $2.7$}
    {Воспользоваться алгоритмом из теоремы $2.2$}
    \begin{enumerate}
        \item $abaa$, для этого слова $|abaa| = 4$:
              \begin{enumerate}
                  \item $T_0 = \{S\}$
                  \item $T_1 = T_0 \cup \{aAS, a\}$
                  \item $T_2 = T_1 \cup \{aAa, abaS, aSSS\}$
                  \item $T_3 = T_2 \cup \{abaa\}$ На этом шаге мы увидели, что в множестве $T_3$ есть нужная цепочка $abaa$, дальше расписывать для наших задач не имеет смысла, данное слово принадлежит языку, порождаемому грамматикой $G$
              \end{enumerate}
        \item $abbb$ для него $|abbb| = 4$, значит, что множества до $T_3$ будут совпадать. Возьмём $T_0, T_1, T_2$ из предыдущего пункта:
              \begin{enumerate}
                  \item $T_3 = T_2 \cup \{abaa, aSSa, aaSS, aSaS\}$
                  \item $T_4 = T_3 \cup\{aaSa, aSaa, aaaS\}$
                  \item $T_5 = T_4 \cup \{aaaa\}$
                  \item $T_6 = T_5$
              \end{enumerate}
              алгоритм закончен, в множестве $T_5$ не нашлось слова $abbb$, значит слово не принадлежит языку, порождаемому грамматикой $G$
        \item $baaba$, для него $|baaba| = 5$:
              \begin{enumerate}
                  \item $T_0 = \{S\}$
                  \item $T_1 = T_0 \cup \{aAS, a\}$
                  \item $T_2 = T_1 \cup \{aSbAS, abaS, aSSS, aAaAS, aAa\}$
                  \item $T_3 = T_2 \cup \{aabAS, aSbAa, aaSS, aSaS, aSSa, aAaAa, abaa\}$
                  \item $T_4 = T_3 \cup \{aabAa, aaaS, aaSa, aSaa\}$
                  \item $T_5 = T_4 \cup \{aaaa\}$
                  \item $T_6 = T_5$
              \end{enumerate}
              алгоритм закончил свою работу, так как $baaba \notin T_5$ слово не принадлежит языку, порождаемому грамматикой $G$
    \end{enumerate}

\end{solution}

\section*{Упражнение \romannumeralCaps 1-2.5}
\begin{solution}
    {Контекстно-свободная или регулярная грамматика $G$, теорема $2.2$.}
    {Ответ  на вопрос, можно ли улучшить оценку для $m$.}
    {}

\end{solution}
\end{document}